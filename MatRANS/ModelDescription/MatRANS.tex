% Define the document class
\documentclass[12pt]{article}
\bibliographystyle{jfm}

% Lines defining margin
%\topmargin -13mm
%\textwidth 163mm % A4 paper
%\textheight 245mm % 245
%\oddsidemargin 0mm
%\pagestyle{empty}

% Additional packages
\usepackage{graphicx}
\usepackage{verbatim}
\usepackage[mathscr]{eucal}
%\usepackage{algorithmic}
%\usepackage{algorithm}
\usepackage{natbib}
%\usepackage{textcomp}
%\usepackage{scriptsize}

% New commands
%\newcommand{\figscale}{0.7} % Scale to use the figure
\newcommand{\superscript}[1]{\ensuremath{^{#1}}} % Used to put superscripts in text
\newcommand{\Matlab}{{\sc Matlab}\superscript{\scriptsize\textregistered}} % Prints Matlab using small caps font and registered symbol
\newcommand{\Mathematica}{{{\it Mathematica}\texttrademark}} 
\newcommand{\ie}{{\it i.e.\ }}
\newcommand{\eg}{{\it e.g.\ }}
\newcommand{\figpath}{/home/drf/Figures} % General path to figures
\newcommand{\bibpath}{/home/drfu/Latex/Papers/BibTeX} % Path to references                               

\begin{document}

% Heading information
\title{MatRANS: A Matlab-based RANS + $k$-$\omega$ model for turbulent
  boundary layer and sediment transport simulations} \author{David
  R. Fuhrman} \date{\today} \maketitle
%\thispagestyle{empty}

%\begin{abstract}
%  \noindent This is the abstract.
%\end{abstract}


\section{Introduction}

This note describes a Matlab-based implementation solving the
horizontal component of the incompressible Navier-Stokes equations,
combined with the two-equation $k$-$\omega$ turbulence closure model
of \cite{Wilcox2006,Wilcox2008}.  The hydrodynamic model is
additionally coupled with a gradient diffusion model for suspended
sediment calculations, subject to oscillatory and/or steady forcing.
Bed load sediment transport can also be calcuated using various bed
load formulae.  The model is intended to be applicable for simple
steady current and oscillatory wave boundary layer flows, including
combined wave-current simulations.  Its development began during the
MSc project of \cite{SchloerSterner2009}, under the supervision of the
present author, who has since made numerous additional modifications
to the code.  

By utilizing the familiar Matlab environment, it is hoped that the
code makes state-of-the-art boundary layer simulations easily
accessible for students studying turbulence and/or sediment transport
processes, and that it might be useful for both educational and
research purposes.


\section{Model Description}
\label{sec:ModelDescription}

\subsection{Hydrodynamic and turbulence model}

The model solves simplified versions of the horizontal component of
the incompressible Reynolds-averaged Navier-Stokes (RANS) equations,
combined with the two-equation $k$-$\omega$ turbulence closure model
of \cite{Wilcox2006}.  The considered RANS equation reads:
%
\begin{equation}
\frac{\partial u}{\partial t} = 
-\frac{1}{\rho}\frac{\partial p}{\partial x} % Pressure gradient
+ \nu\frac{\partial^2 u}{\partial y^2} % Viscous term
+ \frac{\partial}{\partial y}\left(\nu_T \frac{\partial u}{\partial
    y}\right) % Reynolds stress
-\underline{
 \left( u\frac{\partial u}{\partial x} + v\frac{\partial u}{\partial
    y} \right) % Convective acceleration
- \frac{2}{3}\frac{\partial k}{\partial x}
}.
\label{eqn:u}
\end{equation}
%
The turbulence model consists of two respective transport equations
for the turbulent kinetic energy (per unit mass)
$k=1/2(\overline{u^{\prime2}} + \overline{v^{\prime2}} +
\overline{w^{\prime2}})$, where the prime superscripted variables
represent turbulent velocity fluctuations, as well as the specific
dissipation rate $\omega$:
%
\begin{eqnarray}
\frac{\partial k}{\partial t} & = &
\nu_T\left(\frac{\partial u}{\partial y}\frac{\partial u}{\partial y}\right)
- \beta^* k \omega 
+ \frac{\partial}{\partial y}\left[
\left(\nu + \sigma^*\frac{k}{\omega} \right) \frac{\partial k}{\partial
y}  \right]
\nonumber
\\
&&-\underline{
 \left( u\frac{\partial k}{\partial x} + v\frac{\partial k}{\partial
    y} \right)
}-
\underline{\underline{\frac{\nu_T}{\sigma_p}N^2}},
\label{eqn:k}
\end{eqnarray}
%
\begin{eqnarray}
\frac{\partial \omega}{\partial t} & = &
\alpha\frac{\omega}{k}\nu_T\left(\frac{\partial u}{\partial y}\frac{\partial u}{\partial y}\right)
- \beta \omega^2
+ \frac{\sigma_d}{\omega}\frac{\partial k}{\partial y}\frac{\partial\omega}{\partial y}
+ \frac{\partial}{\partial y}\left[
\left(\nu + \sigma\frac{k}{\omega} \right) \frac{\partial\omega}{\partial
y}  \right]
\nonumber
\\
&&-\underline{
 \left( u\frac{\partial\omega}{\partial x} + v\frac{\partial\omega}{\partial
    y} \right)
}-\underline{\underline{c_{e\epsilon}N^2}}.
\label{eqn:omega}
\end{eqnarray}
%
The eddy viscosity is defined by
%
\begin{equation}
\nu_T = \frac{k}{\tilde{\omega}},
\hspace{1cm}
\tilde{\omega} = \max \left\{\omega, C_{lim}\frac{\left|\partial
      u/\partial y \right|}{\sqrt{\beta^*}}\right\},
\label{eqn:nuT}
\end{equation}
%
where $C_{lim}=7/8$.  In
(\ref{eqn:omega})
%
\begin{equation}
\sigma_d = \mathscr{H}\left\{\frac{\partial k}{\partial
      x_j}\frac{\partial \omega}{\partial x_j}\right\}\sigma_{do},
\end{equation}
%
where $\mathscr{H}\left\{\cdot\right\}$ is the Heaviside step
function, taking a value of zero when the argument is negative, and a
value of unity otherwise.  In the right hand side of (\ref{eqn:k}) the
first term represents the {\bf production} of turbulent kinetic energy
(the rate at which kinetic energy is tranferred from the mean flow to
the turbulence), the second term represents {\bf dissipation} (the
rate at which turbulent kinetic energy is converted into therman
internal energy), the third (single underlined) term includes both
molecular and turbulent {\bf diffusion}, and the fourth term includes
both horizontal and vertical {\bf convection}.  The final (double
underlined) term additionally incorporates turbulence suppression
effects.  This term is not part of the standard \cite{Wilcox2006}
model, and is discussed in more detail in \S\ref{sec:turbsup}.  The
default model closure coefficients suggested by \cite{Wilcox2006} are:
$\alpha=13/25$, $\beta=\beta_0 f_\beta$, $\beta_0=0.0708$,
$\beta^*=9/100$, $\sigma=1/2$, $\sigma^*=3/5$, $\sigma_{do}=1/8$.
Note that for two-dimensional flows $f_\beta=1$.

The most important differences between this version of the
$k$-$\omega$ model and earlier versions created by Wilcox et al.~are
the addition of a ``cross-diffusion'' term (the term proportional to
$\sigma_d$ in (\ref{eqn:omega})) and a built-in ``stress-limiter''
modification (the quantity proportional to $C_{lim}$ in
(\ref{eqn:nuT})).  The cross-diffusion is added to reduce the model's
sensitivity to the free stream value of $\omega$, whereas the effect
of the stress-limiter is most important for compressible flows
\citep{Wilcox2006}.  \cite{Fuhrmanetal2010} have found that the
stress-limiter can create undesirably large Reynolds number dependence
for steady boundary layers under hydraulically rough conditions, when
used in conjunction with a $k=0$ wall boundary condition.  Hence, in
these conditions it can be switched off by simply setting $C_{lim}=0$,
whereby the eddy viscosity reduces to $\nu_T=k/\omega$.

\subsection{Second-order terms}

In (\ref{eqn:u})--(\ref{eqn:omega}) (and (\ref{eqn:C}), below) the
single-underlined terms are of so-called second-order importance.  For
wave boundary layers the importance of these terms (relative to
leading-order terms) can be shown to scale with $a\kappa=U_{1m}/c$,
where $a=U_{1m}/\omega_w$ is the amplitude of free stream orbital
motion (for a sinusoidal velocity variation of magnitude $U_{1m}$ with
angular frequency $\omega_w$) and $\kappa=2\pi/L$ is the wave number
($L$ being the wave length), with $c=\omega_w/\kappa$ the wave
celerity.  These terms are responsible \eg for the phenomenon of
boundary layer streaming within oscillatory flows under progressive
waves.  Alternatively, for flow in an oscillatory u-tube (having
uniformity in the $x$-direction) $\kappa=0$ (that is $L=\infty$) and
the second-order terms drop out entirely.  These terms may be omitted
within the model simply by setting the \texttt{streaming} flag to
zero.  Additional third-order terms are neglected in the model, and
are not described here.

Assuming constant form wave propagation, all $x$-derivatives appearing
in the second-order terms are actually replaced by time derivatives in
the model via the relation
%
\begin{equation}
\frac{\partial}{\partial x} = - \frac{1}{c}\frac{\partial}{\partial t}.
\end{equation}
%
This is convenient, as it allows potential effects from $x$-variations
(\eg boundary layer streaming) to be incorporated, while only
requiring discretization in the vertical $y$-direction.

The vertical velocity $v$ is approximated by solving the local
continuity equation
%
\begin{equation}
\frac{\partial u}{\partial x} + \frac{\partial v}{\partial y} 
\approx -\frac{1}{c}\frac{\partial u^{(1)}}{\partial t} + \frac{\partial v}{\partial y} 
= 0,
\label{eqn:cont}
\end{equation}
%
where $\partial u^{(1)}/\partial t$ is the leading-order (\ie
non-underlined) part of $\partial u/\partial t$ from (\ref{eqn:u}).
This is justifiable because all terms containing $v$ are already of
second-order importance, hence the remaining errors are at third-order
or higher.


\subsection{Turbulence supression terms}
\label{sec:turbsup}

The final double-underlined terms in (\ref{eqn:k}) and
(\ref{eqn:omega}) describe turbulence suppression, due to density
gradients in the fluid-sediment mixture.  These terms are implemented
analogously to recent $k$-$\epsilon$ modelling undertaken by
\cite{Ruessinketal2009}, where
%
\begin{equation}
N = \sqrt{-\frac{g}{\rho_m}\frac{\partial\rho_m}{\partial y}}
\end{equation}
%
is the so-called Brunt-Vaisala frequency, $\rho_m=s\rho c + \rho(1-c)$
is the density of the fluid-sediment mixture, and $s$ is the relative
density of the sediment.  The default closure coefficients are
$\sigma_p=0.7$ with $c_{3\epsilon}=1$ for $N^2<=0$ and
$c_{3\epsilon}=0$ for $N^2>0$.  These terms can be important for flows
giving rise to high suspended sediment concentrations near the bed.
These may be omitted within the model simply by setting the
\texttt{turbsup} flag equal to zero.


\subsection{Boundary conditions}

The above equations are (typically) solved starting from motionless
initial conditions in a single vertical dimension, subject to the
following boundary conditions.  The bottom boundary is considered a
friction wall, and a no-slip boundary condition is imposed, \ie all
velocity variables are set to zero.  The bottom boundary condition for
$\omega$ is adopted from \cite{Wilcox2006}, where
%
\vspace{-10pt}\begin{equation}
\omega = \frac{U_f^2}{\nu} S_R,
\hspace{1cm}
y=0.
\end{equation}
%
\noindent
The factor $S_R$ is based on the roughness Reynolds number $k_N^+=k_N
U_f/\nu$, where $k_N$ is Nikuradse's equivalent sand grain roughness,
and $U_f=\sqrt{\left|\tau_b\right|/\rho}$ is the instantaneous
friction velocity, according to
%
\begin{eqnarray}
S_R = 
\left\{ 
%\Bigg\{ 
\begin{array}{ll}
\left(\frac{200}{k_N^+}\right)^2, & k_N^+ \leq 5,\\
\frac{K_r}{k_N^+} + \left[\left(\frac{200}{k_N^+}\right)^2 - \frac{K_r}{k_N^+} \right]e^{5-k_N^+}, & k_N^+ > 5,
\end{array} 
\right.
\label{eqn:SR}
\end{eqnarray}
%
where \cite{Wilcox2006} suggests using $K_r=100$.  A frictionless
rigid lid is imposed at the top boundary, whereby vertical derivatives
of all variables ($u$, $k$, $\omega$, and $C$) are set to zero.  Note
that better agreement with standard law of the wall solutions for
steady boundary layers under hydraulically rough conditions (using a
$k=0$ wall boundary condition; see below) has been found by modifying
the rough wall coefficient in (\ref{eqn:SR}) slightly to $K_r=80$, as
shown by \cite{Fuhrmanetal2010}.

Both $k=0$ and $dk/dy=0$ bottom wall boundary conditions are
implemented within the model.  The latter condition, while not
conventionally used, is supported by experimental measurements on
rough beds \citep[][]{Sumeretal2003, Fuhrmanetal2010}.
\cite{Fuhrmanetal2010} have also recently argued that the zero
gradient condition is more generally consistent with near-wall physics
than a $k=0$ condition, as it allows a viscous sublayer to develop
near smooth walls, while conveniently avoiding a forced viscous
sublayer near rough walls.  This can yield significant numerical
advantages, especially for large roughness simulations.  Hence, the
zero-gradient condition is preferable is most cases.  If this
conditions is utilized, then the closure coefficient $K_r=180$ is
recommended \citep{Fuhrmanetal2010}.


\subsection{Sediment transport model}

\subsubsection{Bed load models}

Bed load sediment transport is calculated from the bed shear stress
obtained from the hydrodynamics model.  As options, the widely used
\cite{MeyerPeterMuller1948} bedload formula is included, as are
formulae from \cite{EngelundFredsoe1976}, and \cite{Nielsen1992}.
Other bedload formulas of interest may be easily added, if desired.

The model also incorporates potential effects of a longitudinal bed
slope $S$ on the critical Shields parameter, as described in
\cite{FredsoeDeigaard1992}, see their p.~205.  The specified slope is
that of the bottom, hence if it is positive it will increase the
critical Shields parameter (\ie make sediment more difficult to
transport) when the flow is positive (uphill), while decreasing the
critical parameter when the flow is negative (downhill).  The
adjustment is made dynamically, based on the sign of the instantaneous
bed shear stress.

\subsubsection{Suspended sediment model}

The above hydrodynamic model is also coupled with a
turbulent-diffusion equation for the possible simulation of the
suspended sediment concentration $C$ \citep[see
\eg][p.~238]{FredsoeDeigaard1992}:
%
\begin{equation}
\frac{\partial C}{\partial t} = 
\frac{\partial \left(w_s C\right)}{\partial y}
+ \frac{\partial}{\partial y}\left(\epsilon_s \frac{\partial C}{\partial
    y}\right) % Reynolds stress
-\underline{
 \left( u\frac{\partial C}{\partial x} + v\frac{\partial C}{\partial
    y} \right) % Convective acceleration
},
\label{eqn:C}
\end{equation}
%
where $w_s$ is the settling velocity, and
$\epsilon_s=\beta_s\nu_T+\nu$ is the diffusion coefficient, where
commonly $\beta_s=1$ is used, though sometimes values corresponding up
to $\beta_s=2$ \citep[\eg][]{Ruessinketal2009} are utilized with
success.  The evaluation of the various terms in this equation is the
same as described above.

Regarding the bottom boundary condition for the suspended sediment
concentration $C$, a number of options are provided in the model,
which will not be described in detail here.  The implemented options
include the reference concentration methods of
\cite{EngelundFredsoe1976}, \cite{ZysermanFredsoe1994},
\cite{Einstein1950}, and \cite{ODonoghueWright2004}, as well as the
pickup function of \cite{vanRijn1984}.  The option to prevent
un-physical overloading conditions (\ie where the reference
concentration $C_b=C(y=b)$ is forced to be smaller than the
concentration immediately above) is also included, and is controlled
by the \texttt{iextrap} flag.  If this flag is set to unity, then the
reference concentration is taken as the maximum of that computed using
the selected formula and that extrapolated from the two grid points
nearest the bed.  This option \eg avoids directly forcing the
reference concentration $C_b=0$ during times of flow reversal in
oscillatory flows.

Hindered settling effects may optionally be included within the model
(using the \texttt{Hind\_Set} flag).  If this flag is set to unity,
then the fall velocity $w_s$ depends on the local suspended sediment
concentration according to
%
\begin{equation}
w_s = w_{s0}(1-C)^n.
\end{equation}
%
The exponent $n$ may either be input directly, or computed according
to the expressions of \cite{RichardsonZaki1954}; see
\cite{FredsoeDeigaard1992}, p. 200.  Likewise the base settling
velocity $w_{s0}$ may be either input directly, or calculated
automatically for a given grain diameter $d$ according to (7.9),
(7.10), and (7.12) of \cite{FredsoeDeigaard1992}.

Suspended sediment transport is calculated for each time as
%
\begin{equation}
q_S = \int_b^{h_m} uc dy,
\end{equation}
%
where $y=b$ is the reference level, and $h_m$ is the total height of
the modelled domain.  Typically $b=2d$ is utilized.

%Hence,
%positive slope values are intended to provide a local situation
%similar to waves propagating up a beach.



\subsection{Pressure gradient}

A prescribed pressure gradient in (\ref{eqn:u}) is used to drive the
flow within the model domain.  To obtain a desired free stream
velocity signal $U_0$, this can be implemented generally as
%
\begin{equation}
\frac{1}{\rho}\frac{\partial p}{\partial x} =
\left( \underline{\frac{U_0}{c}}-1 \right)
\frac{\partial U_0}{\partial t} + P_x
-\underline{\underline{\underline{S U_0^2/h}}}
,
\label{eqn:px}
\end{equation}
%
where the underlined term is again of secondary importance \ie it is
$O(a\kappa)$.  Note that a constant pressure gradient $P_x$ is also
added, which can be used \eg to drive steady currents or within
combined wave-current simulations.  For pure wave simulations simply
set $P_x=0$.  Within the model a free stream velocity having the form
of a second-order Stokes signal is utilized, defined according to
%
\begin{equation}
U_0 = U_{1m}\sin(\omega_w t^\prime) - U_{2m}\cos(2\omega_w t^\prime)
\label{eqn:U}
\end{equation}
%
\begin{equation}
\frac{\partial U_0}{\partial t} = 
U_{1m}\omega_w\cos(\omega_w t^\prime)
+ 2U_{2m}\omega_w\sin(2\omega_w t^\prime)
\label{eqn:Ut}
\end{equation}
%
where $t^\prime=t+t_0$, where $t_0$ represents a time shift in the
signal, which is automatically determined to ensure that $U_0(t=0)=0$.
Note that a sinusoidal free stream velocity signal is achieved by
setting $U_{2m}=0$, whereby $t_0=0$.  The more flexible wave form
shape proposed recently by \cite{Abreuetal2010} is also implemented
in the model, though this will not be described in detail here.  This
option allows forcing velocity signals ranging from highly skewed to
highly front-back asymmetric (\eg as commonly found in the surf zone).

For steady current simulations the oscillatory part of the pressure
gradient can be switched off by setting $U_{1m}=U_{2m}=0$.  The
constant pressure gradient term will then be $-P_x=U_f^2/h_m$ once
steady-state conditions are achieved, where $h_m$ is the height of the
model domain, hence it can be prescribed based on a desired friction
velocity.

The final triple-underlined term in (\ref{eqn:px}) represents a
modification of the pressure gradient due to potential
converging-diverging effects induced \eg by a sloping bed, where $S$
is the bed slope, and $h$ is the flow depth (or in the case of a
converging-diverging tunnel, the tunnel half depth).  For convenience,
this is left as a free input parameter, since it may be undesirable to
resolve a given flow depth in its entirety (\ie $h$ is not necessarily
taken as equal to $h_m$).  Such converging-diverging effects can be
important within cross-shore boundary layer and sediment transport
dynamics, as discussed by \cite{Sumeretal1993} and
\cite{Fuhrmanetal2009a,Fuhrmanetal2009b}.  This term may be switched
off by setting the \texttt{iconv} flag to zero.


\subsection{Filtering}

Generally, the model is quite robust and stabil, and does not usually
require any filtering for stability purposes.  However, experience has
shown that the inclusion of the turbulence supression terms can
de-stabilize the $k$-equation, sometimes leading to the development of
significant high-wavenumber noise in the vertical profiles.  For this
reason, optional filtering has been added to the evaluation of time
derivative for each of the governing equations (\ref{eqn:u}),
(\ref{eqn:k}), (\ref{eqn:omega}), and (\ref{eqn:C}).  These,
respectively, utilize three-point filter stencils of the form:
%
\begin{equation}
[\alpha_f\hspace{1cm} 1-2\alpha_f\hspace{1cm} \alpha_f].
\end{equation}
%
Setting $\alpha_f=0$ shuts the associated filter off, whereas setting
$\alpha_f=0.25$ eliminates Nyquist frequency modes \cite[see
  \eg][p.~228]{AbbottMinns1998}.  When turbulence suppression is not
included within a simulation it is suggested to turn off filtering
completely.  Alternatively, when the turbulence suppression terms are
included, it is suggested to set the filter coefficient for $dk/dt$ to
$0.25$ and that on $du/dt$ to $0.05$.  These settings have been found
to significantly reduce noise levels in the model when turbulence
suppression is included.


\subsection{Matlab files}

The model described above is essentially contained within two Matlab
M-files: a main execution script \texttt{\mbox{main\_MatRANS.m}},
which calls the function \texttt{ddt\_MatRANS.m}.  A third script
\texttt{GlobalVars.m} simply declares global variables.  By default,
these files are placed into the \texttt{MatRANS/src/} directory.  This
directory should be added to the the Matlab path \eg using the
\texttt{addpath} command.  Input files for individual simulations are
intended to be set up within a separate \texttt{MatRANS.m} file, which
executes the \texttt{main\_MatRANS.m} script at the end.  The
implementation allows various terms to be easily incorporated or shut
off, as described in some instances above.  For example, the inclusion
of the turbulence model itself is controlled by the \texttt{turb} flag
in \texttt{MatRANS.m} \ie if this flag is set to 0 the simulation will
be laminar, whereas if it is 1 then the turbulence model is used.
Similarly, the inclusion of the second-order $O(a\kappa)$ (single
underline) terms above is controlled by the \texttt{streaming} flag
(\ie for flow in a u-tube, these should be switched off by setting
\texttt{streaming}$=0$).  Finally, the inclusion of the suspended
sediment calculation is controlled by the \texttt{susp} flag.  All
input variables are in SI units.

The evaluation of the time derivatives
(\ref{eqn:u})--(\ref{eqn:omega}) and (\ref{eqn:C}) for a given state
is performed within the the function \texttt{ddt\_MatRANS.m}, making
use of finite difference approximations for the required vertical
spatial derivatives.  This function is utilized by Matlab's stiff ODE
solver \texttt{ode15s}, which provides dynamic time step control, and
is used for the time integration.  The call to this internal function
is made in the \texttt{main\_MatRANS.m} script.

For a given set up, the model can be run by simply executing an
individual \texttt{MatRANS.m} script from the Matlab command line.
After a simulation is completed the important variables are exported
to a file having the name specified by the \texttt{OutFileName}
variable, which can then be easily loaded into Matlab.  An example of
how to load the data, plot time series, and make animations from the
model results is provided in a separate script \texttt{PostProcess.m}.


\section{Provided examples}

A number of examples are provided as part of the model.  These have
been selected to include basic (laminar, smooth turbulent, and rough
turbulent) steady current and oscillatory flows.  These include the
following cases, listed together with relevant references:
%
\begin{itemize}
\item Laminar current
\item Laminar wave boundary layer
\item Laminar wave boundary layer streaming \citep{Longuet-Higgins1953}
\item Smooth turbulent current flow \citep[][]{Fuhrmanetal2010}
\item Rough turbulent current flow \citep[][]{Fuhrmanetal2010}
\item Vanoni suspended sediment distribution in a current
\item Smooth turbulent wave boundary layer \citep[][]{Jensenetal1989}
\item Rough turblent wave boundary layer flow \citep[][]{Jensenetal1989}
\item Rough turblent wave boundary layer streaming
  \citep[][]{HolmedalMyrhaug2009}
\end{itemize}
%
The model is a work in progress, and additional test cases are
presently under consideration!

The individual test cases can be simulated simply by executing the
associated \texttt{MatRANS.m} file, again assuming that the basic
model described above are in the Matlab path.  Comparison against
relevant theory or experiments can be obtained afterwards simply by
executing the associated \texttt{Compare.m} script.



\section{Conclusions}
\label{sec:Conclusions}

This note describes a Matlab implementation of a Reynolds-averaged
Navier-Stokes model, combined with the $k$-$\omega$ turbulence closure
of \cite{Wilcox2006}, which can be used for performing simple (wave
plus current) boundary layer simulations.  The hydrodynamic model is
also coupled with a suspended sediment description, and includes
several features believed to play an important role within cross-shore
wave boundary layer and sediment transport dynamics.  These include
\eg non-sinusoidal wave shape, boundary layer streaming, bed slope
modifications to critical Shields parameter, converging-diverging
effects, hindered settling of sediments, as well as turbulence
suppression due to density gradients.  These various features can
easily be switched on or off, as desired.  It is again hoped that, by
utilizing the Matlab environment, the model will be easily accessible
to students for demonstration of turbulence and sediment tranport
processes, including possible research purposes.


\bibliography{\bibpath/References}

\end{document}
